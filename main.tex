% Settings for TeXShop

%!TEX TS-program = lualatex
%!TEX encoding = UTF-8 Unicode

\documentclass[11pt, a4paper]{article}

\usepackage[margin=1in]{geometry}

\usepackage{graphicx}
\usepackage{pgf}
\usepackage{import}

% Showcase of fontspec and 
\usepackage{fontspec}
\usepackage{sourcesanspro}
\usepackage{pxfonts}
\defaultfontfeatures{Mapping=tex-text}
\setromanfont[Mapping=tex-text]{Palatino}
\setmonofont[Scale=MatchLowercase]{Andale Mono}


%%% Local Variables:
%%% mode: latex
%%% TeX-master: "./main"
%%% End:

% \usepackage{lua-visual-debug}
% \usepackage{tikz}
% \usepackage{pgfplots}


\title{Sample Article}
\author{John Snow}
% \date{}                                           % Activate to display a given date or no date
\newcommand*{\mpl}{\texttt{Matplotlib}\xspace}
\newcommand*{\ink}{\texttt{Inkscape}\xspace}
\newcommand*{\py}{\texttt{Python}\xspace}

\begin{document}
\maketitle

\doublespacing


\section{Short Description}
\label{sec:usage}

This is a template project that generates figures with publication
standard using \LaTeXe{}, \mpl and \ink. The idea is to generate
figure and subplot layout using \py, with minimal manual tweaking of
parameters.

The desired appearance of figures output by this package is:
\begin{itemize}
\item Color and data management by \py{} + \mpl{}
\item Font style and size controlled by \LaTeX{} typesetting
\item Figure size automatically detected by \mpl{} using the current \LaTeX{} style
\end{itemize}

There are also other aesthetic considerations:
\begin{itemize}
\item Minimize the number of font sizes used in a single figure
\item Use all \texttt{sans serif} fonts inside the figure for both annotation \& math
\item Automatic labeling subfigures according to the convention of journal publications
\end{itemize}

The following sections describes the basic usage of the repository. 

\section{Layout of directory}
\label{sec:layout-directory}

The \py{} codes to generate a multi-panel figure depends heavily on
the \LaTeX{} preamble and \texttt{matplotlibrc} settings, therefore, I
prefer to put the configuration scripts together with the plotting
scripts, instead of shipping an stand-alone \py{} package.

The \LaTeX{} files are usually placed on the root of the repository,
while the \py{} scripts and image files are places under separate
sub-folders. A typical layout can be:
\vspace{1em}
\dirtree{%
  .1 / \DTcomment{Root for main \LaTeX{} files}.
  .2 main.tex \DTcomment{Main \LaTeX{} file}.
  .2 .test.tex \DTcomment{Minimal \LaTeX{} file to detect page geometry}.
  .2 header.tex \DTcomment{Common preamble for \LaTeX{} files}.
  .2 preamble\_plot.tex \DTcomment{\textit{Optional} preambles for \mpl{} plots}.
  .2 scripts \DTcomment{\py{} scripts}. 
  .3 helper \DTcomment{Helper module for making plots}.
  .3 plot\_xx \DTcomment{Scripts for plotting}.
  .2 img \DTcomment{Directory for \texttt{pgf} image output}.
  .2 TeX \DTcomment{Directory for additional \LaTeX{} files}.
}

To use the package, you may need to tweak a little bit on the \py{}
scripts. The places you may want to look into are:
\begin{itemize}
\item \verb|header.tex|: Font settings for main \LaTeX{} file
\item \verb|preamble_plot.tex|: Font settings for \texttt{pgf} plots
\item \verb|scripts/helper/tex_helper.py|: Which header and preamble files to look for
\item \verb|scripts/helper/mpl_helper.py|: If need to use individual preamble for \mpl{} plot
\end{itemize}

\section{Generating figures using \texttt{Matplotlib} alone}
\label{sec:matpl-only-figure}

The \py{} scripts inside the \texttt{scripts/} folder needed to be
executed as packages, such as:
\begin{verbatim}
python -m plot_xx.plot_xx_yy
\end{verbatim}
if \verb|plot_xx/plot_xx_yy.py| is the script to generate \texttt{pgf}
plot.

\subsection{Automatic detetion of figure width}
\label{sec:autom-detet-figure}

To make the figure sizes compatible with the width of document page,
the \LaTeX{} macro \verb|\textwidth| is detected automatically by
rendering the \verb|.test.tex| file.


{\allowbreak
  As a result, the relative width of a figure to be generated can be
calculated using \verb|mpl_helper.relsize(r)|, where \texttt{r} is the
relative width of the figure (0$\sim{}$1).
\par}

\subsection{Generating subfigures using relative size}
\label{sec:gener-subf}

The function \verb|mpl_helper.grid_plots| provides an interface like
\verb|pyplot.subplots|, while instead providing the full
\texttt{figsize}, the function uses relative size \texttt{r} and a
length:height ratio \texttt{ratio} to determine the figure size.

\textbf{Note}: the size of \verb|\textwidth| is given using points. In
order to convert it into inches that \mpl{} understands, the
\verb|ppi| value is used. By default \verb|ppi=72|.

Usually in publications, the subfigures are labeled as \textbf{a},
\textbf{b}, \textbf{c}, $\dots$. This is achieved using
\verb|constrained_layout = True| and \verb|Figure.transFigure| in
\mpl{}. A function \verb|mpl_helper.grid_labels| is provided to add
text labels to a series of \verb|matplotlib.axes.Axes| subplot axes
instances. 

\textbf{Note}: the position of the labels are determined using a
relatively simple algorithm. If the desired layout is not achieved, I
recommend to do a manual tweaking providing the \verb|offsets|
parameter to the \verb|mpl_helper.grid_labels| function.

\textbf{Usage}: \verb|helper/__init__.py| wraps up all the helper
functions \& variables. Therefore, to call the functions inside the
plot scripts, simply call:
\begin{verbatim}
from helper import gridplots, grid_labels
\end{verbatim}

\autoref{fig:fig1} is a simple example using
\verb|mpl_helper.grid_plots| to create a plot with 4 sub-panels. The
base font size in the plot is 10 pt (compared with the 12 pt) used in
main text. The \texttt{sans serif} font used is \texttt{Fira
  Sans}. The plot is produced using \verb|example1.py|.

\begin{figure}[htbp]
  \centering
  \mathversion{sansmath}
  \import{img/}{example1.pgf}
  \caption{Example of figure with 4 sub-panels produced by
    \texttt{gridplots}.}
  \label{fig:fig1}
\end{figure}

\autoref{fig:fig2} extends the example to make a figure with
unbalanced subfigures. This is achieved by setting \texttt{span}
parameter in the \texttt{gridplots} function. In addition, the figure
uses a small manual tweaking to set the position of label
\textbf{\textsf{d}}. The plot is produced using \verb|example2.py|.


\begin{figure}[htbp]
  \mathversion{sansmath}
  \centering
  \import{img/}{example2.pgf}
  \caption{Example of imbalanced subfigure using sans font}
  \label{fig:fig2}
\end{figure}

\autoref{fig:fig2a} shows a practical example usually encountered in
publication figures: legend shared by two subfigures. In this case,
the legend is usually moved outside the plots, with the aspect ratio
of both figures kept the same. The plot is produced using
\verb|example2a.py|. The position of the legend is set by using both
the \texttt{bbox\_to\_anchor} and \texttt{loc} keywords. Such case is
usually necessary when the number of items in the legend is more than
4 and the text of each item is also long.

\begin{figure}[htbp]
  \mathversion{sansmath}
  \centering
  \import{img/}{example2a.pgf}
  \caption{Example of subfigure with legend outside}
  \label{fig:fig2a}
\end{figure}

\subsection{Mixing subplot with image}
\label{sec:mixing-subplot-with}

The next sample (~\ref{fig:fig3}) is to mix a line plot with a 2D
plot. Note the original \texttt{pgf} output from \mpl{} uses
\verb|\pgfimage| macro, which is not able to track the image input in
sub-folders. Instead, we should replace it with
\verb|\includegraphics|. This is done by the helper function
\verb|savepgf()| inside \verb|mpl_helper.py|.

\begin{figure}[htbp]
  \mathversion{sansmath}
  \centering
  \import{img/}{example3.pgf}
  \caption{Example of line plot side-by-side with 2D plot}
  \label{fig:fig3}
\end{figure}


\end{document}  
% Local Variables:
% TeX-engine: luatex
% TeX-master: t
% End:

%%% Local Variables:
%%% mode: latex
%%% TeX-master: t
%%% End:
