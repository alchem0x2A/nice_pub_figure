% Settings for TeXShop

%!TEX TS-program = lualatex
%!TEX encoding = UTF-8 Unicode

\documentclass[11pt, a4paper]{article}

\usepackage[margin=1in]{geometry}

\usepackage{graphicx}
\usepackage{pgf}
\usepackage{import}

% Showcase of fontspec and 
\usepackage{fontspec}
\usepackage{sourcesanspro}
\usepackage{pxfonts}
\defaultfontfeatures{Mapping=tex-text}
\setromanfont[Mapping=tex-text]{Palatino}
\setmonofont[Scale=MatchLowercase]{Andale Mono}


%%% Local Variables:
%%% mode: latex
%%% TeX-master: "./main"
%%% End:

\usepackage{lua-visual-debug}
\usepackage{tikz}
\usepackage{pgfplots}
\usepgflibrary{pgfplots.groupplots}

\title{Sample Article}
\author{John Snow}
%\date{}                                           % Activate to display a given date or no date

\begin{document}
\maketitle

Figure~\ref{fig:fig1} shows how to use the \texttt{helper} module to
make a balanced subfigure.

\begin{figure}[htbp]
  \centering
  \import{../img/}{example1.pgf}
  \caption{Example of balanced subfigure}
  \label{fig:fig1}
\end{figure}

An example of imbalanced figure is shown in Figure~\ref{fig:fig2}.

\begin{figure}[htbp]
  \centering
  \import{../img/}{example2.pgf}
  \caption{Example of imbalanced subfigure}
  \label{fig:fig2}
\end{figure}

\input{../img/example2.tex}


% \begin{figure}[htbp]
%   \centering
%   \includegraphics{test.pdf}
%   \caption{A test example of mpl output with pdf}
%   \label{fig:fig1}
% \end{figure}

% In the following section you will see Figure \ref{fig:fig2} produced
% by 2D plot. The rasterized \texttt{pgf} plot is by importing an binary
% image inside.

% For now the workaround is to change all the
% \texttt{\textbackslash{}pgfimage} to
% \texttt{\textbackslash{}includegraphics}
% in the PGF part.


% \begin{figure}[htbp]
%   \centering
%   \import{img/}{test2.pgf}
%   % \i{img/test2.pgf}
%   \caption{A test example of mpl output}
%   \label{fig:fig2}
% \end{figure}

% There is a 3rd example working on the addition of external pdf
% (e.g. generated by inkscape etc).

% \begin{figure}[htbp]
%   \centering
%   \import{img/}{test4.pgf}
%   % \i{img/test2.pgf}
%   \caption{A test example of mpl output}
%   \label{fig:fig4}
% \end{figure}

% \begin{figure}[htbp]
%   \centering
%   \import{img/}{test5.pgf}
%   % \i{img/test2.pgf}
%   \caption{A test example of mpl output}
%   \label{fig:fig4}
% \end{figure}



\end{document}  
% Local Variables:
% TeX-engine: luatex
% TeX-master: t
% End:

%%% Local Variables:
%%% mode: latex
%%% TeX-master: t
%%% End:
